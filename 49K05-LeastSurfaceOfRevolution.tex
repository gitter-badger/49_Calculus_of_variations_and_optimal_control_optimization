\documentclass[12pt]{article}
\usepackage{pmmeta}
\pmcanonicalname{LeastSurfaceOfRevolution}
\pmcreated{2013-03-22 19:12:11}
\pmmodified{2013-03-22 19:12:11}
\pmowner{pahio}{2872}
\pmmodifier{pahio}{2872}
\pmtitle{least surface of revolution}
\pmrecord{6}{42118}
\pmprivacy{1}
\pmauthor{pahio}{2872}
\pmtype{Example}
\pmcomment{trigger rebuild}
\pmclassification{msc}{49K05}
\pmclassification{msc}{53A05}
\pmclassification{msc}{26B15}
%\pmkeywords{calculus of variations}
\pmrelated{MinimalSurface}
\pmrelated{EquationOfCatenaryViaCalculusOfVariations}
\pmrelated{Catenary}
\pmrelated{MinimalSurface2}
\pmrelated{CalculusOfVariations}
\pmrelated{SurfaceOfRevolution2}

% this is the default PlanetMath preamble.  as your knowledge
% of TeX increases, you will probably want to edit this, but
% it should be fine as is for beginners.

% almost certainly you want these
\usepackage{amssymb}
\usepackage{amsmath}
\usepackage{amsfonts}

% used for TeXing text within eps files
%\usepackage{psfrag}
% need this for including graphics (\includegraphics)
%\usepackage{graphicx}
% for neatly defining theorems and propositions
 \usepackage{amsthm}
% making logically defined graphics
%%%\usepackage{xypic}

% there are many more packages, add them here as you need them

% define commands here

\theoremstyle{definition}
\newtheorem*{thmplain}{Theorem}

\begin{document}
The points \,$P_1 = (x_1,\,y_1)$\, and\, $P_2 = (x_2,\,y_2)$\, have to be \PMlinkescapetext{connected} by an arc $c$ such that when it rotates around the $x$-axis, the \PMlinkname{area of the surface of revolution}{SurfaceOfRevolution} formed by it is as small as possible. \\

The area in question, expressed by the path integral
\begin{align}
A \;=\; 2\pi\int_{P_1}^{P_2}\!y\,ds,
\end{align}
along $c$, is to be minimised; i.e. we must minimise
\begin{align}
\int_{P_1}^{P_2}\!y\,ds \;=\; \int_{x_1}^{x_2}\sqrt{1\!+\!y'^2}\,|dx|.
\end{align}


Since the integrand in (2) does not explicitly depend on $x$, the \PMlinkname{Euler--Lagrange differential equation}{EulerLagrangeDifferentialEquation} of the problem, the necessary condition for (2) to give an extremal $c$, reduces to the Beltrami identity
$$y\sqrt{1\!+\!y'^2}-y'\!\!\cdot\!\frac{yy'}{\sqrt{1\!+\!y'^2}} 
\;\equiv\; \frac{y}{\sqrt{1\!+\!y'^2}} \;=\; a,$$
where $a$ is a constant of integration.\, After solving this equation for the derivative $y'$ and separation of variables, we get
$$\pm\frac{dy}{\sqrt{y^2\!-\!a^2}} \;=\; \frac{dx}{a},$$
by integration of which
we choose the new constant of integration $b$ such that\, $x = b$\, when\, $y = a$:
$$\pm\int_a^y\frac{dy}{\sqrt{y^2\!-\!a^2}} \;=\; \int_b^x\frac{dx}{a}$$
We can write two \PMlinkname{equivalent}{Equivalent3} results
$$\ln\frac{y\!+\!\sqrt{y^2\!-\!a^2}}{a} \;=\; +\frac{x\!-\!b}{a}, \qquad 
  \ln\frac{y\!-\!\sqrt{^2\!-\!a^2}}{a} \;=\; -\frac{x\!-\!b}{a},$$
i.e.
$$\frac{y\!+\!\sqrt{y^2\!-\!a^2}}{a} \;=\; e^{+\frac{x-b}{a}}, \qquad 
  \frac{y\!-\!\sqrt{y^2\!-\!a^2}}{a} \;=\; e^{-\frac{x-b}{a}}.$$
Adding these yields
\begin{align}
y \;=\; \frac{a}{2}\!\left(e^{\frac{x-b}{a}}+e^{-\frac{x-b}{a}}\right) \;=\; a\cosh\frac{x\!-\!b}{a}.
\end{align}
From this we see that the extremals $c$ of the problem are catenaries.\, It means that the least surface of revolution in the question is a catenoid.


%%%%%
%%%%%
\end{document}
