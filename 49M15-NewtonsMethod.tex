\documentclass[12pt]{article}
\usepackage{pmmeta}
\pmcanonicalname{NewtonsMethod}
\pmcreated{2013-03-22 11:58:05}
\pmmodified{2013-03-22 11:58:05}
\pmowner{alozano}{2414}
\pmmodifier{alozano}{2414}
\pmtitle{Newton's method}
\pmrecord{28}{30789}
\pmprivacy{1}
\pmauthor{alozano}{2414}
\pmtype{Algorithm}
\pmcomment{trigger rebuild}
\pmclassification{msc}{49M15}
\pmsynonym{Newton-Raphson method}{NewtonsMethod}
%\pmkeywords{divide and average}
%\pmkeywords{Kantorovitch's theorem}
%\pmkeywords{Lipschitz ratio}
%\pmkeywords{finding roots}
\pmrelated{LipschitzCondition}
\pmrelated{KantorovitchsTheorem}
\pmrelated{Derivative}
\pmrelated{Superconvergence}
\pmrelated{FixedPointsOfNormalFunctions}

\endmetadata

\usepackage{amssymb}
\usepackage{amsmath}
\usepackage{amsfonts}
\usepackage{graphicx}
%%%\usepackage{xypic}
\begin{document}
\newcommand{\fvec}{{\mathbf{f}}}
\newcommand{\xpt}{\mathbf{x}}
\newcommand{\dfa}{[\mathbf{D}(\mathbf{a_0})]}
\newcommand{\ao}{\mathbf{a_0}}
\newcommand{\ovec}{\mathbf{0}}
\newcommand{\xvec}{\mathbf{x}}
\newcommand{\bvec}{\mathbf{b}}

Let $\fvec$ be a differentiable function from $\mathbb{R}^n$ to $\mathbb{R}^n$.
Newton's method consists of starting at an $\ao$ for the equation $\fvec(\xpt)=\ovec$. Then the function is linearized at $\ao$ by replacing the increment $\fvec(\xpt)-\fvec(\ao)$ by a linear function of the increment $\dfa(\xpt-\ao)$.

Now we can solve the linear equation $\fvec(\ao)+\dfa(\xpt-\ao)=\ovec$.
Since this is a system of \emph{n} linear equations in \emph{n} unknowns, 
$\dfa(\xpt-\ao) = -\fvec(\ao)$ can be likened to the general linear system  $A\xvec=\bvec$.  

Therefore, if $\dfa$ is invertible, then $\xpt = \ao-\dfa^{-1}\fvec(\ao)$.
By renaming $\xpt$ to $\mathbf{a_1}$, you can reiterate Newton's method to get an $\mathbf{a_2}$. Thus, Newton's method states

$$\mathbf{a}_{n+1} = \mathbf{a}_n - [\mathbf{D}\fvec(\mathbf{a}_n)]^{-1}\fvec(\mathbf{a}_n)$$

Thus we get a series of $\mathbf{a}$'s that we hope will converge to $\xpt$ such that $\fvec(\xpt)=\ovec$.  When we solve an equation of the form $\fvec(\xpt) = \ovec$, we call the solution a \emph{root} of the equation. Thus, Newton's method is used to find roots of nonlinear equations.

Unfortunately, Newton's method does not always converge. There are tests for 
neighborhoods of $\ao$'s where Newton's method will converge however. One such test is Kantorovitch's theorem, which combines what is needed into a concise mathematical equation.

\textbf{Corollary 1: Newton's Method in one dimension} - The above equation is simplified in one dimension to the well-used 

$$a_1 = a_0 - \frac{f(a_0)}{f'(a_0)}$$ 

This intuitively cute equation is pretty much \textbf{the} equation of first year calculus. :)

\textbf{Corollary 2: Finding a square root} - So now that you know the equation, 
you need to know how to \emph{use} it, as it is an algorithm. The construction of the primary equation, of course is the important part. Let's see how you do it if you want to find a square root of a number \emph{b}.

We want to find a number \emph{x} (\emph{x} for unknown), such that $x^2=b$. You might think ``why not find 
a number such that  $x=\sqrt{b}$ ?'' Well, the problem with that approach is that we don't have a value for $\sqrt{b}$, so we'd be right back where we started. However, squaring both sides of the equation to get $x^2 = b$ lets us work with the number we \emph{do} know, $b$.)  Back to $x^2=b$.  With some manipulation, we see this means that $x^2-b=0$ ! Thus we have our $f(x)=0$ scenario.

We can see that $f'(x)=2x$ thus, $f'(a_0)=2a_0$ and $f(a_0)=a_0^2-b$. Now we have all we need to carry out Newton's method. By renaming $x$ to be $a_1$, we have 

$$a_1 = a_0 - \frac{1}{2a_0}(a_0^2-b) = \frac{1}{2}\left(a_0+\frac{b}{a_0}\right)$$.

The equation on the far right is also known as the \emph{divide and average} method, for those who have not learned the full Newton's method, and just want a fast way to find square roots. Let's see how this works out to find the square root of a number like 2:

Let $x^2=2$

$$x^2-2=0=f(x)$$

Thus, by Newton's method,...

$$a_1 = a_0 - \frac{a_0^2-2}{2a_0}$$

All we did was plug in the expressions $f(a_0)$ and $f'(a_0)$ where Newton's method asks for them. Now we have to pick an $a_0$. Hmm, since

$$\sqrt{1}<\sqrt{2}<\sqrt{4}$$
$$1<\sqrt{2}<2$$ 

let's pick a reasonable number between 1 and 2 like 1.5

$$a_1 = 1.5 - \frac{1.5^2-2}{2(1.5)}$$
$$a_1 = 1.41\bar{6}$$ 

Looks like our guess was too high. Let's see what the next iteration says

$$a_2 = 1.41\bar{6} - \frac{1.41\bar{6}^2-2}{2(1.41\bar{6})}$$
$$a_2 = 1.414215686\dots$$ 

getting better =) You can use your calculator to find that

$$\sqrt{2} = 1.414213562\dots$$

Not bad for only two iterations! Of course, the more you iterate, the more decimal places your $a_n$ will be accurate to. By the way, this is also how your calculator/computer finds square roots!

\textbf{Geometric Interpretation:} Consider an arbitrary function $f\colon\mathbb{R}\rightarrow\mathbb{R}$ such as $f(x)=x^2-b$. Say you wanted to find a root of this function. You know that in the neighborhood of $x=a_0$, there is a root (Maybe you used Kantorovitch's theorem or tested and saw that the function changed signs in this neighborhood). We want to use our knowledge $a_0$ to find an $a_1$ that is a better approximation to $x_0$ (in this case, closer to it on the x-axis).

\begin{center}
\input{fig1.pstex_t}
\end{center}


So we know that $x_0\leq a_1\leq a_0$ or in another case $a_0\leq a_1\leq x_0$. What is an efficient algorithm to bridge the gap between $a_0$ and $x_0$ ? Let's look at a tangent line to the graph.

\begin{center}
\input{fig2.pstex_t}
\end{center}


Note that the line intercepts the x-axis between $a_0$ and $x_0$, which is exactly what we want. The slope of this tangent line is $f'(a_0)$ by definition of the derivative at $a_0$, and we know one point on the line is $(a_1,0)$, since that is the x-intercept. That is all we need to find the formula of the line, and solve for $a_1$.

\begin{align*}
y-y_1 &= m ( x - x_1 ) &\\
f(a_0)-0 &= f'(a_0)(a_0-a_1)& \text{Substituting}\\
\frac{f(a_0)}{f'(a_0)}&=a_0-a_1 &\\
-a_1 &= -a_0 + \frac{f(a_0)}{f'(a_0)} & \text{Aesthetic change. Flipped the equation around.}\\
a_1 &= a_0 - \frac{f(a_0)}{f'(a_0)} & \text{Newton's method!}
\end{align*}
%%%%%
%%%%%
%%%%%
\end{document}
