\documentclass[12pt]{article}
\usepackage{pmmeta}
\pmcanonicalname{EquationOfCatenaryViaCalculusOfVariations}
\pmcreated{2013-03-22 19:12:07}
\pmmodified{2013-03-22 19:12:07}
\pmowner{pahio}{2872}
\pmmodifier{pahio}{2872}
\pmtitle{equation of catenary via calculus of variations}
\pmrecord{10}{42117}
\pmprivacy{1}
\pmauthor{pahio}{2872}
\pmtype{Derivation}
\pmcomment{trigger rebuild}
\pmclassification{msc}{49K05}
\pmclassification{msc}{49K22}
\pmclassification{msc}{47A60}
\pmrelated{Catenary}
\pmrelated{CalculusOfVariations}
\pmrelated{LeastSurfaceOfRevolution}

\endmetadata

% this is the default PlanetMath preamble.  as your knowledge
% of TeX increases, you will probably want to edit this, but
% it should be fine as is for beginners.

% almost certainly you want these
\usepackage{amssymb}
\usepackage{amsmath}
\usepackage{amsfonts}

% used for TeXing text within eps files
%\usepackage{psfrag}
% need this for including graphics (\includegraphics)
%\usepackage{graphicx}
% for neatly defining theorems and propositions
 \usepackage{amsthm}
% making logically defined graphics
%%%\usepackage{xypic}

% there are many more packages, add them here as you need them

% define commands here

\theoremstyle{definition}
\newtheorem*{thmplain}{Theorem}

\begin{document}
Using the mechanical principle that the centre of mass \PMlinkescapetext{places} itself as low as possible, determine the equation of the curve formed by a \PMlinkescapetext{flexible homogeneous wire or a thin chain with length} $l$ when supported at its ends in the points \,$P_1 = (x_1,\,y_1)$\, and\, $P_2 = (x_2,\,y_2)$.\\


We have an isoperimetric problem 
\begin{align}
\mbox{to minimise} \quad \int_{P_1}^{P_2}\!y\,ds
\end{align}
under the constraint
\begin{align}
\int_{P_1}^{P_2}\!ds \;=\; l,
\end{align}
where both the path integrals are taken along some curve $c$.\, Using a Lagrange multiplier $\lambda$, the task changes to a free problem
\begin{align}
\int_{P_1}^{P_2}\!(y\!-\!\lambda)\,ds \;=\; \int_{x_1}^{x_2}(y\!-\!\lambda)\sqrt{1\!+\!y'^2}\,|dx| \;=\; \mbox{min}!
\end{align}
(cf. example of calculus of variations).

The \PMlinkname{Euler--Lagrange differential equation}{EulerLagrangeDifferentialEquation}, the necessary condition for (3) to give an extremal $c$, reduces to the Beltrami identity
$$(y\!-\!\lambda)\sqrt{1\!+\!y'^2}-y'\!\cdot\!(y\!-\!\lambda)\!\cdot\!\frac{y'}{\sqrt{1\!+\!y'^2}} 
\;\equiv\; \frac{y\!-\!\lambda}{\sqrt{1\!+\!y'^2}} \;=\; a,$$
where $a$ is a constant of integration.\, After solving this equation for the derivative $y'$ and separation of variables, we get
$$\pm\frac{dy}{\sqrt{(y\!-\!\lambda)^2\!-\!a^2}} \;=\; \frac{dx}{a}$$
which may become clearer by notating\, $y\!-\!\lambda := u$;\, then by integrating
$$\pm\frac{du}{\sqrt{u^2\!-\!a^2}} \;=\; \frac{dx}{a}$$
we choose the new constant of integration $b$ such that\, $x = b$\, when\, $u = a$:
$$\pm\int_a^u\frac{du}{\sqrt{u^2\!-\!a^2}} \;=\; \int_b^x\frac{dx}{a}$$
We can write two \PMlinkname{equivalent}{Equivalent3} results
$$\ln\frac{u\!+\!\sqrt{u^2\!-\!a^2}}{a} \;=\; +\frac{x\!-\!b}{a}, \qquad 
  \ln\frac{u\!-\!\sqrt{u^2\!-\!a^2}}{a} \;=\; -\frac{x\!-\!b}{a},$$
i.e.
$$\frac{u\!+\!\sqrt{u^2\!-\!a^2}}{a} \;=\; e^{+\frac{x-b}{a}}, \qquad 
  \frac{u\!-\!\sqrt{u^2\!-\!a^2}}{a} \;=\; e^{-\frac{x-b}{a}}.$$
Adding these allows to eliminate the square roots and to obtain
$$u \;=\; \frac{a}{2}\!\left(e^{\frac{x-b}{a}}+e^{-\frac{x-b}{a}}\right),$$
or
\begin{align}
 y\!-\!\lambda \;=\; a\cosh\frac{x\!-\!b}{a}.
\end{align}
This is the sought form of the equation of the chain curve.\, The constants $\lambda,\,a,\,b$ can then be determined for putting the curve to pass through the given points $P_1$ and $P_2$.

\begin{thebibliography}{8}
\bibitem{lindelof}{\sc E. Lindel\"of}: {\em Differentiali- ja integralilasku
ja sen sovellutukset IV. Johdatus variatiolaskuun}.\, Mercatorin Kirjapaino Osakeyhti\"o, Helsinki (1946).
\end{thebibliography}
%%%%%
%%%%%
\end{document}
