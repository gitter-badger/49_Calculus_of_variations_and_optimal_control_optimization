\documentclass[12pt]{article}
\usepackage{pmmeta}
\pmcanonicalname{ProofOfArithmeticgeometricMeansInequalityUsingLagrangeMultipliers}
\pmcreated{2013-03-22 15:25:53}
\pmmodified{2013-03-22 15:25:53}
\pmowner{stevecheng}{10074}
\pmmodifier{stevecheng}{10074}
\pmtitle{proof of arithmetic-geometric means inequality using Lagrange multipliers}
\pmrecord{16}{37278}
\pmprivacy{1}
\pmauthor{stevecheng}{10074}
\pmtype{Example}
\pmcomment{trigger rebuild}
\pmclassification{msc}{49-00}
\pmclassification{msc}{26B12}
%\pmkeywords{Lagrange multiplier}
%\pmkeywords{arithmetic-geometric means inequality}
\pmrelated{ArithmeticGeometricMeansInequality}
\pmrelated{UsingJensensInequalityToProveTheArithmeticGeometricHarmonicMeansInequality}

\endmetadata

% this is the default PlanetMath preamble.  as your knowledge
% of TeX increases, you will probably want to edit this, but
% it should be fine as is for beginners.

% almost certainly you want these
\usepackage{amssymb}
\usepackage{amsmath}
\usepackage{amsfonts}

% used for TeXing text within eps files
\usepackage{psfrag}
% need this for including graphics (\includegraphics)
\usepackage{graphicx}
% for neatly defining theorems and propositions
%\usepackage{amsthm}
% making logically defined graphics
%%%\usepackage{xypic}

% there are many more packages, add them here as you need them
\usepackage{enumerate}

% define commands here
\newcommand{\real}{\mathbb{R}}
\newcommand{\rat}{\mathbb{Q}}
\newcommand{\nat}{\mathbb{N}}

\providecommand{\abs}[1]{\lvert#1\rvert}
\providecommand{\absW}[1]{\left\lvert#1\right\rvert}
\providecommand{\absB}[1]{\Bigl\lvert#1\Bigr\rvert}
\providecommand{\norm}[1]{\lVert#1\rVert}
\providecommand{\normW}[1]{\left\lVert#1\right\rVert}
\providecommand{\normB}[1]{\Bigl\lVert#1\Bigr\rVert}
\providecommand{\defnterm}[1]{\emph{#1}}

\DeclareMathOperator{\D}{D}
\begin{document}
As an interesting example of the Lagrange multiplier method,
we employ it to prove the arithmetic-geometric means inequality:
\[
\sqrt[n]{x_1 \dotsm x_n} \leq \frac{x_1 + \dotsb + x_n}{n}\,, \quad x_i \geq 0\,,
\]
with equality if and only if all the $x_i$ are equal.

To begin with, define $f\colon \real_{\geq 0}^n \to \real_{\geq 0}$ by $f(x) = (x_1 \dotsm x_n)^{1/n}$.

Fix $a > 0$ (the arithmetic mean),
and consider the set $M = \{ x \in \real^n : x_i \geq 0 \text{ for all $i$, } \sum_i x_i = na \}$.
This is clearly a closed and bounded set in $\real^n$, and hence is compact.
Therefore $f$ on $M$ has both a maximum and minimum.

$M$ is almost a manifold (a differentiable surface), except that it has a boundary consisting
of points with $x_i = 0$ for some $i$.
Clearly $f$ attains the minimum zero on this boundary,
and on the ``interior'' of $M$,
that is $M' = \{ x \in \real^n : x_i > 0 \text{ for all $i$, } \sum_i x_i = na \}$,
$f$ is positive. Now $M'$ is a manifold, and the maximum of $f$ on $M$
is evidently a local maximum on $M'$.  Hence
the Lagrange multiplier method may be applied.

\begin{center}
\includegraphics[scale=0.75]{agm}
\end{center}

We have the constraint\footnote{Although $g$ does not constrain that $x_i > 0$,
this does not really matter --- for those who are worried about such
things, a rigorous way to escape this annoyance is to simply
define $f(x) = 0$ to be zero whenever $x_i \leq 0$ for some $i$.
Clearly the new $f$ cannot have a maximum at such points,
so we may simply ignore the points with $x_i \leq 0$ when solving the system.
And for the same reason, the fact that $f$ fails to be differentiable at the boundary
points is immaterial.} $0 = g(x) = \sum_i x_i - na$
under which we are to maximize $f$.

We compute:
\begin{align*}
\frac{f(x)}{nx_i} = \frac{\partial f}{ \partial x_i} = \lambda \, \frac{\partial g}{\partial x_i} = \lambda\,.
\end{align*}
If we take the product, $i = 1, \dotsc, n$, of the extreme left- and right- hand sides of this equation,
we obtain
\[
\frac{1}{n^n} = \frac{f(x)^n}{n^n x_1 \dotsm x_n} = \lambda^n
\]
from which it follows that $\lambda = 1/n$ ($\lambda = -1/n$ is obviously inadmissible).
Then substituting back, we get $f(x) = x_i$, which implies that the $x_i$ are all equal to $a$,
The value of $f$ at this point is $a$.

Since we have obtained a unique solution to the Lagrange multiplier system,
this unique solution must be the unique maximum point of $f$ on $M$.
So $f(x) \leq a$ amongst all numbers $x_i$ whose arithmetic mean is $a$,
with equality occurring if and only if all the $x_i$ are equal to $a$. 

The case $a = 0$, which was not included in the above analysis, is trivial.

%\section*{Another inequality}
%The expression $f(x)/x_i$ that had appeared looks suspiciously like the derivative of the logarithm.  In fact, there is a problem in information theory that can be solved by applying Lagrange multipliers to the exact same set $M$, with $a = 1/n$:
%
%The uniform probability distribution
%maximizes the \PMlinkname{entropy}{ShannonsTheoremEntropy}
%among all discrete probability distributions $\{ x_i \}$ on $n$ points.
%
%Here we are to maximize the entropy function $h(x) = -\sum_i x_i \log x_i$.
%(On the boundary of $M$, $h$ is zero; on the interior of $M$, $h$ is positive.)  
%Applying Lagrange multipliers again,
%we have
%\[
%- \log (x_i) - 1= \frac{\partial h}{\partial x_i} = \lambda \frac{\partial g}{\partial x_i} = \lambda\,,
%\]
%and it follows that $x_i = e^{-\lambda - 1} = 1/n$ gives the unique maximum point of $h$.

\section*{Proof of concavity}
The question of whether the point obtained from solving Lagrange system
is actually a maximum for $f$ was taken care of by our preliminary compactness argument. Another popular way to determine whether a given stationary point
is a local maximum or minimum is by studying the Hessian of the function
at that point.
For the sake of illustrating the relevant techniques,
we will prove again that $x_i = a$ is a local maximum for $f$,
this time by showing that the Hessian is negative definite.

Actually, it turns out to be not much
harder to show that $f$ is weakly concave \emph{everywhere}
on $\real_{\geq 0}^n$, not just on $M$.
A plausibility argument for this fact
is that the graph of $t \mapsto \sqrt[n]{t}$
is concave, and since $f$ also involves an $n$th root,
it should be concave too.  We will prove concavity of $f$ by
showing that the Hessian of $f$ is negative semi-definite.

Since $M$ is a \emph{convex}\footnote{
If $M$ is not convex, then the arguments that follow will not work.
In general, a prescription to determine whether a critical point
is a local minimum or maximum can be found
in tests for local extrema in Lagrange multiplier method.} set, the restriction
of $f$ to $M$ will also be weakly concave;
then we can conclude that
the critical point $x_i = a$ on $M$
must be a \emph{global} minimum on $M$.

We begin by calculating second-order derivatives:
\begin{align*}
\frac{\partial^2 f}{\partial x_j \partial x_i} &= \frac{\partial}{\partial x_j} \left( \frac{f(x)}{nx_i} \right) 
=
\frac{1}{nx_i} \frac{\partial f}{\partial x_j} + f(x) \frac{\partial}{\partial x_j} \frac{1}{nx_i} \\
&= \frac{f(x)}{n^2 x_i x_j} - \delta_{ij} \frac{f(x)}{n x_i^2}\,.
\end{align*}
(The symbol $\delta_{ij}$ is the Kronecker delta.)
Thus the Hessian matrix is:
\begin{align*}
\D^2 f(x) = \frac{f(x)}{n^2} \left[ \frac{1}{x_i x_j} \right]_{ij} - \frac{f(x)}{n^2} 
\begin{bmatrix}
\frac{n}{x_1^2} \\
& \ddots \\
& & \frac{n}{x_n^2}
\end{bmatrix}
\end{align*}
Now if we had actual numbers to substitute for the $x_i$,
then we can go to a computer and ask if the matrix is negative definite or not.
But we want to prove global concavity, so we have to employ some clever
algebra, and work directly from the definition.

Let $v = (v_1, \dotsc, v_n)$ be a test vector.
Negative semi-definiteness means $v \, \D^2 f(x) \, v^{\mathrm{T}} \leq 0$ for all $v$.
So let us write out:
\begin{align*}
v \D^2 f(x) v^{\mathrm{T}} &= \frac{f(x)}{n^2} \left( \sum_{i,j} \frac{v_i}{x_i} \frac{v_j}{x_j} - n \sum_i \frac{v_i^2}{x_i^2} \right) \\
&= \frac{f(x)}{n^2} \left( \left( \sum_i \frac{v_i}{x_i} \right)^2 - n \sum_i \left(\frac{v_i}{x_i}\right)^2 \right)\,.
\end{align*}
We would be done if we can prove
\begin{equation}\label{cauchy}
\left( \sum_i w_i \right)^2 \leq n \sum_i w_i^2\,, \quad w_i = \frac{v_i}{x_i}\,.
\end{equation}
But look, this is just the Cauchy-Schwarz inequality,
concerning the dot product of the vectors $(w_1, \dotsc, w_n)$
and $(1, \dotsc, 1)$!  
So $\D^2 f(x)$ is negative semi-definite everywhere,
i.e. $f$ is weakly concave.

Moreover, the case of equality in the Cauchy-Schwarz inequality \eqref{cauchy}
only occurs when $(w_1, \dotsc, w_n)$ is parallel to $(1, \dotsc, 1)$,
i.e. $v_i = \mu x_i$ for some scalar $\mu$.
Notice that $(1, \dotsc, 1)$ is the normal vector to
the tangent space of $M$,
so $(1, \dotsc, 1) \cdot v = \mu \sum_i x_i \neq 0$
unless $\mu = 0$.
This means, equality never holds in \eqref{cauchy}
if $v \neq 0$ is \emph{restricted to the tangent space} of $M$.
In turn, this means that $f$ is \emph{strictly} concave
on the convex set $M$, and
so $x_i = a$ is a strict global minimum of $f$ on $M$.

\section*{Proof of inequality by symmetry argument}
%After analyzing the concavity of $f$ completely,
%the author cannot resist giving two other proofs
%of the arithmetic-geometric means inequality,
%based on concavity arguments only, without using Lagrange multipliers at all.
%
%(Actually, techniques based on convexity and concavity are good to know for another reason: 
%they can be applied for infinite-dimensional optimization, where compactness 
%arguments may not be readily available.)
%
%\paragraph{First proof.} We use the fact that the logarithmic function is concave, and Jensen's inequality
%for concave functions
%\[
%\frac{\log x_1 + \dotsc + \log x_n}{n} \leq \log \left( \frac{x_1 + \dotsc + x_n}{n} \right)
%\]
%with equality if and only if the $x_i$ are all equal.
%Taking exponential of both sides yields the result.
%

Observe that the maximum point $x_i = a$ is pleasantly symmetric.
We can show that the solution has to be symmetric,
by exploiting the symmetry of the function $f$ and the set $M$.

We have already calculated that $f$ is strictly concave on $M$;
this was independent of the Lagrange multiplier method.
Since $f$ is strictly concave on a closed convex set,
it has a unique maximum there; call it $x$.

Pick any indices $i$ and $j$, ranging from $1$ to $n$, and form
the linear transformation $T$, which switches the $i$th and $j$th coordinates
of its argument.

By definition, $f(x) \geq f(\xi)$ for all $\xi \in M$.
But $f \circ T = f$, so this means $f(Tx) \geq f(T\xi)$ for all $\xi \in M$.
But $M$ is mapped to itself by the transformation $T$,
so this is the same as saying that $f(Tx) \geq f(\xi)$ for all $\xi \in M$.
But the global maximum is unique, so $Tx = x$,
that is, $x_i = x_j$.

Note that the argument in this last proof
is extremely general --- it takes the form:
if $x$ is the unique maximum of $f \colon M \to \real$,
and $T$ is any symmetry of $M$ such that $f \circ T = f$,
then $Tx = x$.  This kind of symmetry argument was used to great effect
by Jakob Steiner in his famous attempted proof of the isoperimetric inequality:
among all closed curves of a given arc length, the circle maximizes
the area enclosed.  However, Steiner did not realize that
the assumption that there is a unique maximizing curve
has to be proved.  Actually, that is the hardest part of the proof ---
if the assumption is known, then one may easily calculate using the \emph{Lagrange multiplier method}
that the maximizing curve must be a circle!

%%%%%
%%%%%
\end{document}
