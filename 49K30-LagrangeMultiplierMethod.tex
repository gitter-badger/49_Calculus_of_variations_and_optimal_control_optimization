\documentclass[12pt]{article}
\usepackage{pmmeta}
\pmcanonicalname{LagrangeMultiplierMethod}
\pmcreated{2013-03-22 12:25:10}
\pmmodified{2013-03-22 12:25:10}
\pmowner{cvalente}{11260}
\pmmodifier{cvalente}{11260}
\pmtitle{Lagrange multiplier method}
\pmrecord{10}{32352}
\pmprivacy{1}
\pmauthor{cvalente}{11260}
\pmtype{Definition}
\pmcomment{trigger rebuild}
\pmclassification{msc}{49K30}
%\pmkeywords{constraint}
%\pmkeywords{extrema}
\pmrelated{ExampleOfCalculusOfVariations}
\pmrelated{IsoperimetricProblem}
\pmdefines{Lagrange multiplier}

\usepackage{graphicx}
%%%\usepackage{xypic} 
\usepackage{bbm}
\newcommand{\Z}{\mathbbmss{Z}}
\newcommand{\C}{\mathbbmss{C}}
\newcommand{\R}{\mathbbmss{R}}
\newcommand{\Q}{\mathbbmss{Q}}
\newcommand{\mathbb}[1]{\mathbbmss{#1}}
\newcommand{\figura}[1]{\begin{center}\includegraphics{#1}\end{center}}
\newcommand{\figuraex}[2]{\begin{center}\includegraphics[#2]{#1}\end{center}}
\begin{document}
The Lagrange multiplier method is used when one needs to find the extreme or stationary points of a function on a set which is a subset of the domain.

{\bf Method}

Suppose that $f(\mathbf{x})$ and $g_{i}(\mathbf{x}), i=1,...,m$ ($\mathbf{x}\in \R^n$) are differentiable functions that map $\R^n \mapsto \R$, and we want to solve 
$$\min f(\mathbf{x}), \max f(\mathbf{x})\quad\mbox{such that}\quad g_{i}(\mathbf{x})=0,\quad i=1,\ldots,m$$

By a calculus theorem, if the constaints are independent, the gradient of $f$, $\nabla f$, must satisfy the following equation at the stationary points:

$$\nabla f = \sum_{i=1}^{m} \lambda_{i} \nabla g_{i}$$

The constraints are said to be independent iff all the gradients of each constraint are linearly independent, that is:

$\left \{\nabla g_{1}(\mathbf{x}), \ldots, \nabla g_{m}(\mathbf{x})\right \}$ is a set of linearly independent vectors on all points where the constraints are verified.


Note that this is equivalent to finding the stationary points of:

$$f(\mathbf{x})-\sum_{i=1}^{m} \lambda_{i}( g_{i}(\mathbf{x}))$$

for $\mathbf{x}$ in the domain and the \emph{Lagrange multipliers} $\lambda_{i}$ without restrictions.

After finding those points, one applies the $g_i$ constraints to get the actual stationary points subject to the constraints.
%%%%%
%%%%%
\end{document}
