\documentclass[12pt]{article}
\usepackage{pmmeta}
\pmcanonicalname{PlateausProblem}
\pmcreated{2013-03-22 13:38:13}
\pmmodified{2013-03-22 13:38:13}
\pmowner{paolini}{1187}
\pmmodifier{paolini}{1187}
\pmtitle{Plateau's Problem}
\pmrecord{12}{34286}
\pmprivacy{1}
\pmauthor{paolini}{1187}
\pmtype{Topic}
\pmcomment{trigger rebuild}
\pmclassification{msc}{49Q05}
\pmrelated{MinimalSurface2}
\pmrelated{LeastSurfaceOfRevolution}
\pmdefines{minimal surface}

% this is the default PlanetMath preamble.  as your knowledge
% of TeX increases, you will probably want to edit this, but
% it should be fine as is for beginners.

% almost certainly you want these
\usepackage{amssymb}
\usepackage{amsmath}
\usepackage{amsfonts}

% used for TeXing text within eps files
%\usepackage{psfrag}
% need this for including graphics (\includegraphics)
%\usepackage{graphicx}
% for neatly defining theorems and propositions
%\usepackage{amsthm}
% making logically defined graphics
%%%\usepackage{xypic}

% there are many more packages, add them here as you need them

% define commands here
\begin{document}
The ``\PMlinkescapetext{Plateau}'s Problem'' is the problem of finding the surface with minimal area among all surfaces which have the same prescribed boundary.

This problem is named after the Belgian physicist Joseph \PMlinkescapetext{Plateau} (1801-1883) who experimented with soap films. As a matter of fact if you take a wire (which represents a closed curve in three-dimensional space) and dip it in a solution of soapy water, you obtain a soapy surface which has the wire as boundary.
It turns out that this surface has the minimal area among all surfaces with the same boundary, so the soap film is a solution to the \PMlinkescapetext{Plateau}'s Problem.

Jesse Douglas (1897-1965) solved the problem by proving the existence of such minimal surfaces. The solution to the problem is achieved by finding an harmonic and conformal parameterization of the surface.

The extension of the problem to higher dimensions (i.e. for $k$-dimensional surfaces in $n$-dimensional space) turns out to be much more difficult to study.
Moreover while the solutions to the original problem are always regular it turns out that the solutions to the extended problem may have singularities if 
$n\ge 8$.
To solve the extended problem the theory of perimeters (De Giorgi) for
boundaries and the theory of rectifiable
currents (Federer and Fleming) has been developed.
%%%%%
%%%%%
\end{document}
