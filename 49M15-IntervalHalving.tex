\documentclass[12pt]{article}
\usepackage{pmmeta}
\pmcanonicalname{IntervalHalving}
\pmcreated{2013-03-22 14:20:02}
\pmmodified{2013-03-22 14:20:02}
\pmowner{mathcam}{2727}
\pmmodifier{mathcam}{2727}
\pmtitle{interval halving}
\pmrecord{12}{35804}
\pmprivacy{1}
\pmauthor{mathcam}{2727}
\pmtype{Algorithm}
\pmcomment{trigger rebuild}
\pmclassification{msc}{49M15}
\pmsynonym{bisection algorithm}{IntervalHalving}

\endmetadata

% this is the default PlanetMath preamble.  as your knowledge
% of TeX increases, you will probably want to edit this, but
% it should be fine as is for beginners.

% almost certainly you want these
\usepackage{amssymb}
\usepackage{amsmath}
\usepackage{amsfonts}

% used for TeXing text within eps files
%\usepackage{psfrag}
% need this for including graphics (\includegraphics)
%\usepackage{graphicx}
% for neatly defining theorems and propositions
%\usepackage{amsthm}
% making logically defined graphics
%%%\usepackage{xypic}

% there are many more packages, add them here as you need them

% define commands here
\newcommand{\Lindent}{0.4in}
\newenvironment{Lalgorithm}[4]{
\textbf{Algorithm} \textsc{#1}\texttt{(#2)}\newline
\textit{Input}: #3\newline
\textit{Output}: #4

}{}
\newenvironment{Lfloatalgorithm}[6][h]{
\begin{figure}[#1]
\caption{#2}
\begin{Lalgorithm}{#3}{#4}{#5}{#6}
}{
\end{Lalgorithm}
\end{figure}
}
\newcommand{\Lgets}{\ensuremath{\gets}}
\newcommand{\Lgroup}[1]{\textbf{begin}\\\hspace*{\Lindent}\parbox{\textwidth}{#1}\\\textbf{end}}

\newcommand{\Lif}[2]{\textbf{if} #1 \textbf{then}\\\hspace*{\Lindent}\parbox{\textwidth}{#2}}

\newcommand{\Lwhile}[2]{\textbf{while} #1 \\\hspace*{\Lindent}\parbox{\textwidth}{#2}}

\newcommand{\Lelse}[1]{\textbf{else}\\\hspace*{\Lindent}\parbox{\textwidth}{#1}}
\newcommand{\Lelseif}[2]{\textbf{else if} #1 \textbf{then}\\\hspace*{\Lindent}\parbox{\textwidth}{#2}}
\begin{document}
Interval halving is an efficient method for solving equations. The requirements for using this method are that we have an equation $f(x)=0$ where $f(x)$ is a continuous function, and two values $x_1$ and $x_2$ such that $f(x_1)f(x_2)<0$. 
Since $f(x_1)$ and $f(x_2)$ have opposite signs, we know by the intermediate value theorem that there exists a solution $\hat{x}$ and that $x_1 \leq \hat{x}\leq x_2$, and with only $n+1$ function evaluations we can find a shorter interval of length  $\epsilon=|x_1-x_2|2^{-n}$ that contains $\hat{x}$. If we take the solution $x_{s}=(x_1+x_2)/2$ we get an error $ |x_s-\hat{x}|< \epsilon$.


\begin{Lalgorithm}{IntervalHalving}{$x_1,x_2,f(x),\epsilon $}{$x_1, x_2, f(x)$ as above. $\epsilon$ is the length of the desired interval}{A solution to the equation $f(x)=0$ that lies in an interval of length $<\epsilon$ . Requires $f(x_1)f(x_2)<0$}

$start \gets min(x_1,x_2)$ \\
$end \gets max(x_1,x_2)$ \\
$middle \gets (start+end)/2$ \\

\Lwhile{$end-start< \epsilon$}{
\Lgroup{
\Lif{$f(middle)f(start)<0$}{$end \gets middle$}
\Lelse{$start \gets middle$}
$middle \gets (start+end)/2$
}
}
$x_s \gets (start+end)/2$ // the solution
\end{Lalgorithm}



The algorithm works by taking an interval $[x_1,x_2]$ and dividing it into two intervals of equal size. Look at a point in the middle, $x_m$. If the function changes sign in the interval $[x_1,x_m]$, that is $f(x_1)f(x_m)<0$, then by the intermediate value theorem we have a solution in this interval. If not, then the solution is in the interval $[x_m,x_2]$. Repeat this until you get a sufficiently smal interval.



Interval halving is very useful in that it only requires the function to be continuous. More efficient methods like Newton's method require that the function is  differentiable and that we have to have a good starting point.
Interval halving has linear convergence while Newton's method has quadratic convergence.
%%%%%
%%%%%
\end{document}
