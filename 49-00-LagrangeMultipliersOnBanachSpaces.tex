\documentclass[12pt]{article}
\usepackage{pmmeta}
\pmcanonicalname{LagrangeMultipliersOnBanachSpaces}
\pmcreated{2013-03-22 15:28:30}
\pmmodified{2013-03-22 15:28:30}
\pmowner{stevecheng}{10074}
\pmmodifier{stevecheng}{10074}
\pmtitle{Lagrange multipliers on Banach spaces}
\pmrecord{5}{37329}
\pmprivacy{1}
\pmauthor{stevecheng}{10074}
\pmtype{Theorem}
\pmcomment{trigger rebuild}
\pmclassification{msc}{49-00}
\pmclassification{msc}{49K35}

\endmetadata

\usepackage{amssymb}
\usepackage{amsmath}
\usepackage{amsfonts}
%\usepackage{amsthm}
\usepackage{enumerate}

% used for TeXing text within eps files
%\usepackage{psfrag}
% need this for including graphics (\includegraphics)
%\usepackage{graphicx}
% making logically defined graphics
%%%\usepackage{xypic}

% define commands here
\newcommand{\complex}{\mathbb{C}}
\newcommand{\real}{\mathbb{R}}
\newcommand{\rat}{\mathbb{Q}}
\newcommand{\nat}{\mathbb{N}}

\providecommand{\abs}[1]{\lvert#1\rvert}
\providecommand{\absW}[1]{\left\lvert#1\right\rvert}
\providecommand{\absB}[1]{\Bigl\lvert#1\Bigr\rvert}
\providecommand{\norm}[1]{\lVert#1\rVert}
\providecommand{\normW}[1]{\left\lVert#1\right\rVert}
\providecommand{\normB}[1]{\Bigl\lVert#1\Bigr\rVert}
\providecommand{\defnterm}[1]{\emph{#1}}

\DeclareMathOperator{\D}{D}
\DeclareMathOperator{\linspan}{span}
\begin{document}
Let $U$ be open in a real Banach space $X$,
and $Y$ be another real Banach space.
Let $f\colon U \to \real$ and $g\colon U \to Y$
be continuously differentiable functions.

Suppose that $a$ is a minimum or maximum point of $f$
on $M = \{ x \in U : g(x) = 0 \}$,
and the Fr\'echet derivative $\D g(a)\colon X \to Y$
is surjective.  Then there exists a Lagrange multiplier vector
$\lambda \in Y^*$
such
that
\[
\D f(a) = \D g(a)^* \lambda = \lambda \circ \D g(a)\,.
\]
(The function $\D g(a)^*\colon Y^* \to X^*$ denotes 
the pullback or adjoint by $\D g(a)$ on the continuous duals,
defined by the second equality.)

If $X$ and $Y$ are finite-dimensional, writing out the above
equation in matrix form shows that $\lambda$ really
is the usual Lagrange multiplier vector. The condition
that $\D g(a)$ is surjective means that $\D g(a)$
must have full rank as a matrix.

\begin{thebibliography}{3}
\bibitem{Zeidler}
Eberhard Zeidler. {\it Applied functional analysis: main principles and their applications}. Springer-Verlag, 1995.
\end{thebibliography}
%%%%%
%%%%%
\end{document}
