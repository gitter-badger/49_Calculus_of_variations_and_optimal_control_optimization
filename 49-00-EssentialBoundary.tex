\documentclass[12pt]{article}
\usepackage{pmmeta}
\pmcanonicalname{EssentialBoundary}
\pmcreated{2013-03-22 15:01:54}
\pmmodified{2013-03-22 15:01:54}
\pmowner{paolini}{1187}
\pmmodifier{paolini}{1187}
\pmtitle{essential boundary}
\pmrecord{9}{36742}
\pmprivacy{1}
\pmauthor{paolini}{1187}
\pmtype{Definition}
\pmcomment{trigger rebuild}
\pmclassification{msc}{49-00}

% this is the default PlanetMath preamble.  as your knowledge
% of TeX increases, you will probably want to edit this, but
% it should be fine as is for beginners.

% almost certainly you want these
\usepackage{amssymb}
\usepackage{amsmath}
\usepackage{amsfonts}

% used for TeXing text within eps files
%\usepackage{psfrag}
% need this for including graphics (\includegraphics)
%\usepackage{graphicx}
% for neatly defining theorems and propositions
%\usepackage{amsthm}
% making logically defined graphics
%%%\usepackage{xypic}

% there are many more packages, add them here as you need them

% define commands here
\newcommand{\R}{\mathbf R}

\begin{document}
Let $E\subset \R^n$ be a measurable set. We define the \emph{essential boundary} of $E$ as
\[
  \partial^* E := \{x\in\R^n\colon 0 < | E\cap B_\rho(x)| < |B_\rho(x)|,\quad \forall \rho>0\}
\]
where $|\cdot|$ is the Lebesgue measure.

Compare the definition of $\partial^* E$ with the definition of the topological boundary $\partial E$ which can be written as
\[
  \partial E = \{ x \in \R^n \colon \emptyset \subsetneq E\cap B_\rho(x) \subsetneq B_\rho(x),\quad \forall \rho>0\}.
\]
Hence one clearly has $\partial^* E\subset \partial E$.

Notice that the essential boundary does not depend on the Lebesgue 
representative of the set $E$, in the sense that if $|E\triangle F|=0$ then 
$\partial^* E = \partial ^* F$. For example if $E=\mathbf Q^n\subset \R^n$ is 
the set of points with rational coordinates, one has $\partial^* E=\emptyset$ 
while $\partial E=\R^n$.

Nevertheless one can easily prove that $\partial^*E$ is always a closed set (in the usual sense).

%%%%%
%%%%%
\end{document}
